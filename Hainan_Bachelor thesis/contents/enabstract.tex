% !Mode:: "TeX:UTF-8"
\newpage
\cfoot{}
% \addcontentsline{toc}{section}{摘\ \ 要}%加入目录
\centerline{\zihao{3}{Abstract}}
\vspace {12pt}

 With the increasing and deepening of computer applications, network security issues are more complex and prominent. It is not only related to our personal privacy, but also to commercial interests and even national security. The existing security technologies can guarantee a certain security of the network environment, but because of the endless stream of attacks, existing security technologies cannot guarantee absolute security. how to monitor network attacks and make corresponding defenses has become an important issue that needs to be solved in today's network security.
 
 Data is a product of the network age. The traditional data mining-based intrusion detection model relies entirely on data mining algorithms for learning data samples in tagged data sets. The correctness and purity of the data sample markers is critical to building an effective intrusion detection system. But the amount of data in the network is huge, and it takes a lot of money to get pure sample data. Therefore, it is important to find an intrusion detection method that requires less high data sets.
 
 Based on the above background, this paper conducts intrusion detection research based on clustering technology. This experiment is based on the Windows platform, using the Anaconda integrated environment to process and analyze the KDD99 dataset. In the case of unknown data sample categories, the sample is classified by calculating the distance between the samples, and finally the clustering result is obtained.

\vspace {12pt}
\noindent\textbf{[Key Words]:}
data;network security;clustering
