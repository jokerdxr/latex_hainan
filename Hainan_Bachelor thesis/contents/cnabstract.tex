% !Mode:: "TeX:UTF-8"
\newpage
\cfoot{}
% \addcontentsline{toc}{section}{摘\ \ 要}%加入目录
\centerline{\zihao{3}{摘\ \ \ \ 要}} % 三号,加粗,居中

\vspace {12pt}  %空1行

随着计算机应用日趋广泛和深入,网络安全问题也更加复杂和突出,它不仅仅关系到我们个人的隐私,更关系到商业利益乃至国家安全。现有的各种安全技术可以保证网路环境一定的安全性,但由于攻击手段层出不穷,现有安全技术也无法保证绝对安全。如何监控网络攻击并做出相应的防御已成为当今网络安全需要解决的重要问题。
% \par
数据是网络时代的产物,传统的基于数据挖掘的入侵检测模型完全依赖于数据挖掘算法对已标记数据集中数据样本的学习。数据样本标记的正确性和纯度对于构建有效的入侵检测系统至关重要。但网络中的数据量巨大,想要得到纯净的样本数据代价极大。因此寻找一种对数据集要求不那么高的入侵检测方法至关重要。

基于以上背景,本文基于聚类技术进行入侵检测研究。本实验是在Windows平台,采用Anaconda集成环境,对KDD99数据集进行处理分析,在未知数据样本类别的情况下,通过计算样本彼此间的距离来估计样本所属类别,最终得出聚类结果。



\vspace {12pt}

\noindent{\textbf{[关键词]:} % 临时取消行缩进,加粗
数据;网络安全;聚类