\section{基于聚类算法的入侵检测研究}

\subsection{k-means算法介绍}
k-means算法是一种简单的划分聚类算法,它通过计算样本之间的距离大小将样本集划分为k个簇。聚类目标是使同一个簇中的相关对象尽可能相互“接近”,而不同簇中的对象尽可能地“远离”。

\IncMargin{1em} % 使得行号不向外突出 
\begin{algorithm} 
	\caption{kmeans}
	\label{algo:1} 
	\KwData{样本数据集 $D$, 聚类簇数 $k$} 
	\KwResult{聚类集合} 
	$r\leftarrow t$\; $\Delta B^{\ast}\leftarrow -\infty$\; \While{$\Delta B\leq \Delta B^{\ast}$ and $r\leq T$}{$Q\leftarrow\arg\max_{Q\geq 0}\Delta B^{Q}_{t,r}(I_{t-1},B_{t-1})$\; $\Delta B\leftarrow \Delta B^{Q}_{t,r}(I_{t-1},B_{t-1})/(r-t+1)$\; \If{$\Delta B\geq \Delta B^{\ast}$}{$Q^{\ast}\leftarrow Q$\; $\Delta B^{\ast}\leftarrow \Delta B$\;} $r\leftarrow r+1$\;} 
\end{algorithm}
\DecMargin{1em}

从Algorithm \ref{algo:1}中可得 k-means 算法有以下缺点:
\begin{enumerate}  
	\item 聚类中心的k数量需要提前给出,但实际运用中,对于给定数据确定k值十分困难。很多时候我们不知道应该将这些数据集划分成几类最佳。
	\item k-means算法的聚类中心是随机选择的,选择不好的初始聚类中心,可能导致完全不同的聚类结果。
\end{enumerate}
\subsection{最佳k值选择}


\setcounter{table}{0}
\setcounter{figure}{0}
\setcounter{equation}{0}