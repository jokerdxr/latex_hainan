% !Mode:: "TeX:UTF-8"
\section{绪论}

\subsection{选题的背景及意义}
\subsubsection{选题的背景}
由于计算机技术的不断更新,人们已经完全进入了互联网时代。与此同时,网络安全问题日益严重。互联网的广泛开放和移动支付的普及使得一些重要领域受到了越来越多的入侵攻击。网络安全不仅是一个技术问题,而且已成为全球主要的信息安全问题。

最大的勒索病毒比特币勒索软件在2018年袭击了世界,造成了无法估量的损失。尽管网络安全问题得到广泛的关注,但此类事件并未减少。因此,构建数据分析模型使安全人员能够及时检测入侵十分重要,性能好的数据分析模型不仅预测准确,还能节省预测的时间成本。

\subsubsection{选题的意义}
目前世界上入对侵检测的研究涉及众多学科,如统计学、数据挖掘、机器学习等。为了获得更好的入侵特征,本课题基于聚类分析,从网络信息安全领域的先验知识入手,提取那些反映出网络异常行为的特征,然后使用恰当的算法对处理后的数据进行挖掘。

本课题的研究重点是基于网络的无监督异常检测系统的数据分析方法。由于入侵检测要分析的数据量巨大,数据特征复杂、维度较高,因此在聚类分析前要对数据进行大量处理,以便于观察和分析聚类结果。

\subsection{国内外发展状况及其研究方向}

入侵检测系统,简称IDS(Intrusion Detection System),是网络空间安全中的一个重要问题。 它是一个实时监控网络或网络内部的系统。 一旦发现攻击尝试或攻击,IDS将发出警告并提示以确保网络安全。 由于传统防火墙大多使用静态防御并且缺乏实时警告,因此它们也无法攻击深度攻击。而 IDS可以实时响应入侵。

现如今,国外的一些研究机构对入侵检测的研究水平较高,普渡大学、加州大学的Davis分校等在此领域处于国际领先高度。国外的一些知名厂商如Cisco等对于此的研究也很深入。对于IDS的研究国内起步的较晚,但发展迅速,许多国内厂商已经转向入侵检测领域,并且还推出了自己的网络安全产品,如中科网络的“天眼”入侵检测系统,启明星辰的SkyBell和绿盟网络入侵检测。 然而,由于当前入侵检测技术中的各种缺陷,并且各种类型的攻击不断更新,误报率和误报率都很高。 因此,需要进一步提高入侵检测的准确性。就目前而言,模式匹配技术仍然是大多数成熟商家用作IDS的主要技术。

\subsection{本课题的研究概况}
\subsubsection{研究内容}
\subsubsection{研究重点及难点}


\subsection{论文的结构及内容安排}
第一章为绪论。介绍了课题的背景和本课题重要性,除此之外,基于国内外对网络入侵检测的研究现状,简要介绍了研究的内容、难点以及本课题的主要工作。

第二章为本课题使用的相关技术的概述。介绍了开发环境、开发语言以及配套技术,为本课题做准备工作。

第三章为数据集预处理。介绍了分析数据集的方法并通过特征工程执行特征提取和特征降维。

第四章为聚类算法设计。本章分析k-means算法并对其进行优化以实现更好的聚类结果。

第五章为总结。指出在数据预处理中忽视的细节和需要进行改进的地方,还有对本次实验的总结归纳。


\setcounter{table}{0}
\setcounter{figure}{0}
\setcounter{equation}{0}








